\documentclass[12pt,letterpaper]{hmcpset}
\usepackage[margin=1in]{geometry}
\usepackage{graphicx}
\usepackage{amsmath}

% info for header block in upper right hand corner
\name{ }
\class{Math 172}
\assignment{Assignment 11}
\duedate{Tuesday, April 24, 2018}

\newcommand{\pn}[1]{\left( #1 \right)}
\newcommand{\abs}[1]{\left| #1 \right|}
\newcommand{\bk}[1]{\left[ #1 \right]}

\newcommand{\vb}{\mathbf{v}}
\newcommand{\ub}{\mathbf{u}}
\newcommand{\fx}{f \left( x \right) =}
\newcommand*\LH{\ensuremath{\overset{\kern2pt L'H}{=}}}
\renewcommand{\labelenumi}{{(\alph{enumi})}}

\newcommand{\Ff}{\mathbb{F}}
\newcommand{\Zz}{\mathbb{Z}}
\newcommand{\Rr}{\mathbb{R}}
\newcommand{\Qq}{\mathbb{Q}}
\newcommand{\Cc}{\mathbb{C}}
\newcommand{\Aut}[1]{\text{Aut} \left(#1\right) }

\begin{document}

\problemlist{Problems 1 - 6 on Sakai}

\begin{problem}[1.]
  If $W$ and $V$ are algebraic sets, show that $W \cap V$ is an algebraic set.
\end{problem}
\begin{solution}
  \vfill
\end{solution}
\newpage
\begin{problem}[2.]
  If $W$ and $V$ are algebraic sets, show that $W \cup V$ is an algebraic set.
\end{problem}
\begin{solution}
  \vfill
\end{solution}
\newpage
\begin{problem}[3.]
  Show that the set $D = \{(x,x) : x \in \Rr \text{ but } x \ne 1\}$ is NOT an algebraic set. (Hint: if $f \in K[x,y]$ vanishes on $D$, what can be said about $f(1,1)$?)
\end{problem}
\begin{solution}
  \vfill
\end{solution}
\newpage
\begin{problem}[4.]
  Given any monomial order, show that for nonzero polynomials $f$ and $g$, that multidegree$(fg)$=multidegree$(f)$+multidegree$(g)$.
  Also, multidegree$(f+g) \le max$ {multidegree$(f)$, multidegree$(g)$ } with equality if multidegree$(f)$ is not equal multidegree$(g)$.
\end{problem}
\begin{solution}
  \vfill
\end{solution}
\newpage
\begin{problem}[5.]
  The grevlex monomial order is defined by comparing by the total degree first, then comparing exponents of the last indeterminate $x_n$ but reversing the outcome (so the monomial with smaller exponent is larger in the ordering), followed (in case of a tie) by a similar comparison of $x_{n-1}$, and so forth ending with x1.  Show that graded lex order and grevlex order are the same for monomials with 1 and 2 indeterminates, but find an example of two monomials (involving up to 3 indeterminates) for which lex and grevlex order those monomials differently.
\end{problem}
\begin{solution}
  \vfill
\end{solution}
\newpage
\begin{problem}[6.]
  Perform polynomial division on $(x^3 y + y)$ by $f=(xy+2x)$ and  $g=(x^3+1)$ to get a remainder term, and show that the remainder depends on the order of $f$ and $g$ in the polynomial division.
\end{problem}

\end{document}
