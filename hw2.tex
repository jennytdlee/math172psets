\documentclass[12pt,letterpaper]{hmcpset}
\usepackage[margin=1in]{geometry}
\usepackage{graphicx}
\usepackage{amsmath}

% info for header block in upper right hand corner
\name{ }
\class{Math 172}
\assignment{Assignment 2}
\duedate{Tuesday, January 30, 2016}

\newcommand{\pn}[1]{\left( #1 \right)}
\newcommand{\abs}[1]{\left| #1 \right|}
\newcommand{\bk}[1]{\left[ #1 \right]}

\newcommand{\vb}{\mathbf{v}}
\newcommand{\ub}{\mathbf{u}}
\newcommand{\fx}{f \left( x \right) =}
\newcommand*\LH{\ensuremath{\overset{\kern2pt L'H}{=}}}
\renewcommand{\labelenumi}{{(\alph{enumi})}}

\newcommand{\Zz}{\mathbb{Z}}
\newcommand{\Rr}{\mathbb{R}}
\newcommand{\Qq}{\mathbb{Q}}
\newcommand{\Cc}{\mathbb{C}}

\begin{document}

\problemlist{7.4 ( 10, 19 ) 7.5 ( R3 ) 8.2 ( 2, 3 ) 9.1 ( 2, 4 )}

\begin{problem}[7.4.10]
	Assume $R$ is commutative. Prove that if $P$ is a prime ideal of $R$ and $P$ contains no zero divisors, then $R$ is an integral domain.
\end{problem}

\begin{solution}
\vfill
\end{solution}
\newpage

\begin{problem}[7.4.19]
	Let $R$ be a finite commutative ring with identity. Prove that every prime ideal of $R$ is a maximal ideal.
\end{problem}

\begin{solution}
\vfill
\end{solution}
\newpage

\begin{problem}[READ ONLY 7.5.3]
	Let $F$ be a field. Prove that $F$ contains a unique smallest subfield $F_0$ and that $F_0$ is isomorphic to either $\Qq$ or $\Zz/p\Zz$ for some prime $p$ ($F_0$ is called the \emph{prime subfield} of $F$). [See Exercise 26, Section 3.]
\end{problem}

\begin{problem}[8.2.2]
  Prove that any two nonzero elements of a PID have a least common multiple (cf. Exercise 11, Section 1).
\end{problem}
\begin{solution}
\vfill
\end{solution}
\newpage

\begin{problem}[8.2.3]
	Prove that a quotient of a PID by a prime ideal is again a PID.
\end{problem}
\begin{solution}
	\vfill
\end{solution}
\newpage

\begin{problem}[9.1.3]
	If $R$ is a commutative ring and $x_1, x_2, \dots, x_n$ are independent variables over $R$, prove that $R[x_{\pi(1)}, x_{\pi(2)}, \dots, x_{\pi(n)}]$ is isomorphic to $R[x_1,x_2,\dots,x_n]$ for any permutation $\pi$ of $\{1,2,\dots,n\}.$
\end{problem}
\begin{solution}
	\vfill
\end{solution}
\newpage

\begin{problem}[9.1.4]
  Prove that the ideals $(x)$ and $(x,y)$ are prime ideals in $\Qq[x,y]$ but only the latter is a maximal ideal.
\end{problem}
\begin{solution}
	\vfill
\end{solution}

\end{document}
