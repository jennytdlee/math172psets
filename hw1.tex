\documentclass[12pt,letterpaper]{hmcpset}
\usepackage[margin=1in]{geometry}
\usepackage{graphicx}
\usepackage{amsmath}

% info for header block in upper right hand corner
\name{ }
\class{Math 172}
\assignment{Assignment 1}
\duedate{Tuesday, January 23, 2016}

\newcommand{\pn}[1]{\left( #1 \right)}
\newcommand{\abs}[1]{\left| #1 \right|}
\newcommand{\bk}[1]{\left[ #1 \right]}

\newcommand{\vb}{\mathbf{v}}
\newcommand{\ub}{\mathbf{u}}
\newcommand{\fx}{f \left( x \right) =}
\newcommand*\LH{\ensuremath{\overset{\kern2pt L'H}{=}}}
\renewcommand{\labelenumi}{{(\alph{enumi})}}

\newcommand{\Zz}{\mathbb{Z}}
\newcommand{\Rr}{\mathbb{R}}
\newcommand{\Qq}{\mathbb{Q}}
\newcommand{\Cc}{\mathbb{C}}

\begin{document}

\problemlist{7.3 ( 1, 10, 19, 26, 28 ) 7.4 ( 10 ) and Problem A}

\begin{problem}[7.3.1]
	Let $R$ be a ring with identity and $1 \ne 0$. Prove that the rings $2\Zz$ and $3\Zz$ are not isomorphic.
\end{problem}

\begin{solution}
\vfill
\end{solution}
\newpage

\begin{problem}[7.3.10]
	Decide which of the following ideals are ideals of the ring $\Zz[x]$:
	\begin{enumerate}
	    \item
	        the set of all polynomials
	        whose constant term is a
	        multiple of $3$
	    \item
	        the set of all polynomials
	        whose coefficient of
	        $x^2$ is a multiple of $3$
	    \item
	        the set of all polynomials
	        whose constant term, coefficient
	        of $x$ and coefficient of $x^2$
	        are zero
	    \item
	        $\Zz[x^2]$ (i.e., the polynomials
	        in which only even powers of $x$
	        appear)
	    \item
	        the set of polynomials whose
	        coefficients sum to zero
	    \item
	        the set of polynomials $p(x)$
	        such that $p'(0) = 0$, where
	        $p'(x)$ is the usual
	        first derivative of $p(x)$
	        with respect to $x$.
	\end{enumerate}
\end{problem}

\begin{solution}
\vfill
\end{solution}
\newpage

\begin{problem}[7.3.19]
	Prove that if $I_1 \subseteq I_2 \subseteq \dots $ are ideals of $R$ then $\bigcup_{n=1}^\infty I_n$ is an ideal of $R$.
\end{problem}

\begin{solution}
\vfill
\end{solution}
\newpage


\begin{problem}[7.3.26]
The \emph{characteristic}
of a ring $R$ is the smallest
positive integer $n$ such that
\begin{equation*}
    \underbrace{
        1+1+\ldots+1=0
    }_{n \text{ times}}
\end{equation*}
in $R$; if no such integer exists
the characteristic of $R$ is said
to be $0$. For example, $\Zz/n\Zz$
is a ring of characteristic $n$
for each positive integer $n$ and
$\Zz$ is a ring of characteristic
$0$.
\begin{enumerate}
    \item
        Prove that the map
        $\Zz \to R$ defined by
        \begin{equation*}
            k \mapsto
            \begin{cases}
                1 + 1 + \ldots + 1  \; (k \text{ times})
                        & \text{if } k>0 \\
                0
                        & \text{if } k=0 \\
                -1 - 1 - \ldots - 1  \; (-k \text{ times})
                        & \text{if } k<0 \\
            \end{cases}
        \end{equation*}
        is a ring homomorphism whose kernel is
        $n\Zz$, where $n$ is the characteristic
        of $R$. (This explains the use of the
        terminology ``characteristic 0'' instead
        of the archaic phrase
        ``characteristic $\infty$'' for rings
        in which no sum of 1s is zero.)
    \item
        Determine the characteristics
        of the rings $\Qq$,
        $\Zz[x]$, $\Zz/n\Zz[x]$.
    \item
        Prove that if $p$ is a prime
        and if $R$ is a commutative
        ring of characteristic $p$,
        then
        \begin{equation*}
            (a+b)^p = a^p + b^p
        \end{equation*}
        for all $a,b \in R$.
\end{enumerate}
\end{problem}
\begin{solution}
\vfill
\end{solution}
\newpage

\begin{problem}[7.3.28]
	Prove that an integral domain has characteristic $p$, where $p$ is either prime or 0 (cf. Exercise 26).
\end{problem}
\begin{solution}
	\vfill
\end{solution}
\newpage

\begin{problem}[7.4.10]
	Assume $R$ is commutative. Prove that if $P$ is a prime ideal of $R$ and $P$ contains no zero divisors then $R$ is an integral domain.
\end{problem}
\begin{solution}
	\vfill
\end{solution}
\newpage

\begin{problem}[{\bf Problem A}]
	\begin{enumerate}
		\item[(i)]
		Let $R$ be an integral domain. Prove that the units in $R[x]$ are precisely the constant polynomials $p(x) = u$ where $u$ is a unit in $R$.
		\item[(ii)]
		On the other hand, show that $p(x) = 1+2x$ is a unit in $R[x]$, where $R = \Zz/4\Zz$.
	\end{enumerate}
\end{problem}
\begin{solution}
	\vfill
\end{solution}
\end{document}
